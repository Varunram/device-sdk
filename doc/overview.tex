%%--------------------------------------------------
%% Overview
\chapter{Embedded Device SDK}

Atonomi's Embedded Device SDK is a software library providing mechanisms to communicate
with the Atonomi network. Atonomi provides a network solution which tackles
the question of how to establish identity, trust, and reputation of IoT devices.
The blockchain-based approach allows companies to securely exchange services
and data to, from, and involving these devices.

The Embedded Device SDK provides a small-footprint code implementation of
messaging routines for communications between an IoT device and the Atonomi
Identity Registration Network (IRN). Aimed at embedded systems, this library
is intended to be easy to use, integrate, and deploy, and provides support
for vast numbers of SoCs and operating environments, especially those which
leverage ARM Cortex-M or Cortex-A IP cores.

The SDK supports three service endpoints: Activation, Validation, and Reputation.


\section{Terms}
The following terms are used throughout this document and associated source code.

\begin{itemize}
	\item A \textit{Protected Session (PS)} is a solution developed by CENTRI
		Technology which provides secure, encrypted communications between
		devices. Atonomi leverages this solution to secure sensitive
		data.
	\item A \textit{message} is a plaintext representation of some form of
		device-identifying information. These are secured directly via
		CENTRI Protected Sessions.
	\item The term \textit{envelope} is a container for data that is to be
		protected via a Protected Session. Envelopes are encrypted, fully
		secure, and contain no plaintext \textit{messages}. Plaintext
		\textit{messages} can be recovered, however, but only after
		the \textit{envelope} is decrypted (\textit{i.e.}, opened).
	\item Several bytes are prepended to the protected \textit{envelope} to
		form the Atonomi \textit{packet}. These \textit{packets} comprise
		the core messaging protocol between devices and Atonomi servers.
	\item The term \textit{cross-signing} is a piece of data used by or
		pertaining to some device that has been signed by another device
		using that other device's public/private keypair.
\end{itemize}


\section{Device Requirements}
\begin{itemize}
	\item Developers are required to implement a callback to obtain random data from
a Hardware Random Number Generator.
	\item The SDK functions use up to fourty-four hundred (4400 bytes) of stack space.
		Developers are responsible for ensuring this space is available
		in order to prevent a stack overflow or stack-heap collision.
	\item errno-compatible declarations must be available. Developers may either
		include errno.h from a platform's libc, or instead use the provided
		atmi_errno.h file. The SDK is careful to restrict itself to those
		values which are consistent across all major UNIX platforms (Linux, BSD,
		Solaris, AIX, IRIX, etc.), so the source of errno declarations
		that is most convenient should be preferred.
\end{itemize}

\textbf{Note:} Any modifications to the SDK code, in whole or in part, may
result in a failure to successfully communicate with Atonomi servers.


\section{Public Repositories \&{} Project Layout}
The SDK is made available via a GitHub-hosted git repository
\footnote{\url{https://github.com/atonomi/device-sdk}},
where it can easily be cloned or otherwise downloaded for use.

In attempt to maximize ease of use, the SDK consists of a
single-file C implementation, which lends itself to being quickly and
easily integrated into any new or existing codebases. This exposes
the API which developers can use to construct packetized messages for
the Atonomi network.

Atonomi is using CENTRI's Protected Sessions to serve as its protocol's
cryptographic backend, which is a component
\textbf{not} accounted for by the C source alone. Due to licensing
requirements, that component instead must be provided as a prebuilt
static library and is supplied by its vendor, CENTRI Technology.
However, builds for multiple architectures are available and can be
found within the \texttt{lib/} subdirectory. See table \ref{t:libps_arches}
for more details about supported platforms.

\begin{table}[h]
\centering
\begin{tabular}{l l l}
	\textbf{ISA Family} & \textbf{Core Family} & \textbf{Provided Static Library} \\
	\hline
	\texttt{x86_64} & Intel/AMD (64-bit,SSE2)    & \texttt{libps-centri-x64-W.X.Y.Z.a} \\
	\texttt{armv6m} & ARM Cortex-M0/M0+/M1 only  & \texttt{libps-centri-armv6m-W.X.Y.Z.a} \\
	\texttt{armv7m} & ARM Cortex-M3-compatible   & \texttt{libps-centri-armv7m-W.X.Y.Z.a} \\
	\texttt{armv7a} & ARM Cortex-A (32-bit)      & \texttt{libps-centri-armv7a-W.X.Y.Z.a} \\
\end{tabular}
\caption{Machine architectures currently supported by CENTRI Protected Sessions.
	Note \texttt{W}, \texttt{X}, \texttt{Y}, and \texttt{Z} indicate
	arbitrary numeric values which collectively denote the library's
	release version.}
\label{t:libps_arches}
\end{table}

Taken together, the C source and an appropriate CENTRI PS library are
all that are required to get up and running.


\section{Endpoints \&{} HTTP}
The SDK allows one to pack and unpack messages into a secure bytestream
for transmission to and from Atonomi servers. Due to not every device
necessarily having a means of connecting directly to the Internet, no
means of performing this transfer are included in the SDK, and the developer
is responsible for providing the means to do this.

Communications are performed via a simple HTTP-based transaction mechanism
(HTTP version 1.1, specifically). A secure, packed message is provided to
Atonomi servers as the entire body of an HTTP PUT request, and the server
will return the corresponding secure, packed response as the body of an
HTTP 200 (OK) response. No special content encoding is required for the
HTTP transaction, and only two headers are required:
``\texttt{Content-Type:~application/octet-stream}'', and
``\texttt{Content-Length:~DDD}'', where \texttt{DDD} is the numeric length
of the packed request in bytes, expressed in base-10 decimal form.

Corresponding endpoint URLs for the supported request types are:

\begin{center}
\begin{tabular}{l l}
	\textbf{Action} & \textbf{Endpoint URL} \\
	\hline
	Activation & \texttt{http://device.atonomi.net/activation} \\
	Validation & \texttt{http://device.atonomi.net/validation} \\
	Reputation & \texttt{http://device.atonomi.net/reputation} \\
\end{tabular}
\end{center}


