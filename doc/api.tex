%%--------------------------------------------------
%% API Reference
\chapter{API Reference}

\section{Common Library Context}
An \texttt{atmi_context_t} provides a context to API functions with keying
information. This structure must be populated with public and private keys
for use during the packing or unpacking of messages.

\begin{lstlisting}[name=Context Structure]
typedef struct {
	uint8_t   publicKey[32];
	uint8_t   privateKey[32];
} atmi_context_t;
\end{lstlisting}

The \texttt{atmi_context_t} structure is only needed for read access and
may reside in constant / non-volatile memory if desired. Alternatively,
a developer may prefer to construct this on the stack prior to each use
and explicitly clear it afterwards, minimizing exposure to the private key.
Both methods are acceptable.

\begin{lstlisting}
void ATMIsign_device_id(const atmi_context_t *ctx,
                        atmi_session_t *ssn,
                        uint8_t       idsgn_out[72],
                        const uint8_t devid_in[32]);
\end{lstlisting}

The \texttt{ATMIsign_device_id} function is used to cross-sign another
device's Device ID, \texttt{devid_in}, used to create Device Validation and
Device Reputation requests. See those sections below for further details.


\begin{lstlisting}
extern void ATMI_memrand(void *p, size_t n);
\end{lstlisting}

The CENTRI component of the Atonomi packet requires a source of entropy
in order to create any new packages (i.e. an RNG). Due to a limitation in
how they are currently generated, this is exposed not by function pointer
but by declared symbol that the linker is expected to locate.

The \texttt{ATMI_memrand} symbol has been aliased to an Atonomi-specific
symbol name for improved clarity and must be provided by the developer:
he or she must declare a function with this exact signature below.
The function must have C linkage and should obtain the specified number of
bytes of random entropy from a hardware RNG or other similar source and write
them into the location provided.

\textbf{Note:} A CSPRNG (cryptographically secure PRNG) could also suffice
in some cases. All implementation details and choices are left up to the
developer, whom is expected to understand the ramifications and potential
security impacts resulting from said choice.


\section{Session-based Messaging}
The Atonomi network protocol uses session-based messages, where every
request requires a response to be generated and processed. In order to
accomplish this on the device, some memory must be reserved to store
session data for use in processing the response to a corresponding request.
Note this data is unique to each request packet generated. Once the
response is received for that request, however, the state data may be
discarded.

\begin{lstlisting}[name=Session Structure]
typedef struct {
	uint8_t    state[ATMI_SESSBUF_STATE_SIZE];
	uint8_t    packet[ATMI_SESSBUF_SIZE];
} atmi_session_t;
\end{lstlisting}

The developer is responsible for allocating an \texttt{atmi_session_t}
structure, which provides storage for packet state during a transfer,
as well as a working buffer for operations. All message operations
clobber the contents of this buffer; all packing operations
place their output data in this buffer.



\section{Messages for Device Activation}
\begin{lstlisting}[name=Activation Request and Response Structures]
typedef struct {
	uint8_t  id_requestor[32];
} atmi_act_request_t;

typedef struct {
	int32_t  success;
} atmi_act_response_t;
\end{lstlisting}

The \texttt{id_requestor} field should be populated with the device ID that
corresponds to the device making the request.

The \texttt{success} field returned from the server will be zero if the
request was successful, and negative otherwise.

\pagebreak
\begin{lstlisting}[name=Activation Request Packing and Response Unpacking Functions]
int ATMIpack_act_request(const atmi_context_t *ctx,
                         atmi_session_t *ssn,
                         const atmi_act_request_t *act);

int ATMIunpack_act_response(const atmi_context_t *ctx,
                            atmi_session_t *ssn,
                            const void *pinbuf, size_t nin,
                            atmi_act_response_t *act);
\end{lstlisting}

The pack request and unpack response functions require \texttt{ctx}, a pointer
to the Atonomi library context structure with key data, and \texttt{ssn},
a pointer to a new or current session structure. The pack function will
construct an encrypted message with the data in \texttt{act}. The unpack function
will read \texttt{nin} bytes of packed, encrypted data from \texttt{pinbuf},
decrypt it, and place the response contents in \texttt{act}.



\section{Messages for Device Validation}
\begin{lstlisting}[name=Validation Request and Response Structures]
typedef struct {
	uint8_t  id_requestor[32];
	uint8_t  id_requestor_xsigned[72];
	uint8_t  id_subject[32];
} atmi_val_request_t;

typedef struct {
	int32_t  success;
	int8_t   as_initiator;
	int8_t   as_responder;
	int8_t   as_consumer;
	int8_t   as_provider;
} atmi_val_response_t;
\end{lstlisting}

The \texttt{id_requestor} field should be populated with the device ID that
corresponds to the device making the request. The \texttt{id_subject} field
should be populated with the device ID of the device to validate. The value
of the \texttt{id_requestor} field must be sent to the device associated with
\texttt{id_subject} to be cross-signed; this cross-signed result must be used
to populate the \texttt{id_requestor_xsigned} field. Additionally, the caller
must keep a local copy of this value if a reputation amendment request will be
submitted, since the reputation amendment request message also requires this
value. This cross-signed result is non-constant and must be repeated for
each new Validation request.

The \texttt{success} field returned from the server corresponds to the
subject's reputation score. This will be negative if the request was unsuccessful.

The \texttt{int8_t} fields indicate the current reputation state for the
subject device.

\pagebreak
\begin{lstlisting}[name=Validation Request Packing and Response Unpacking Functions]
int ATMIpack_val_request(const atmi_context_t *ctx,
                         atmi_session_t *ssn,
                         const atmi_val_request_t *val);

int ATMIunpack_val_response(const atmi_context_t *ctx,
                            atmi_session_t *ssn,
                            const void *pinbuf, size_t nin,
                            atmi_val_response_t *val);
\end{lstlisting}

The pack request and unpack response functions require \texttt{ctx}, a pointer
to the Atonomi library context structure with key data, and \texttt{ssn},
a pointer to a new or current session structure. The pack function will
construct an encrypted message with the data in \texttt{val}. The unpack function
will read \texttt{nin} bytes of packed, encrypted data from \texttt{pinbuf},
decrypt it, and place the response contents in \texttt{val}.


\section{Messages for Device Reputation}
\begin{lstlisting}[float=h,name=Reputation Request and Response Structures]
typedef struct {
	uint8_t  id_requestor[32];
	uint8_t  id_requestor_xsigned[72];
	uint8_t  id_subject[32];
	uint8_t  comms_initiator;
	uint8_t  comms_replyreceived;
	uint8_t  comms_successful;
	uint32_t comms_noreplytmout_s;
} atmi_rep_request_t;

typedef struct {
	int32_t  success;
} atmi_rep_response_t;
\end{lstlisting}

The \texttt{id_requestor} field should be populated with the device ID that
corresponds to the device making the request. The \texttt{id_subject} field
should be populated with the device ID of the device to validate. The value
of the \texttt{id_requestor} field must have been sent to the device associated
with \texttt{id_subject} to be cross-signed; this cross-signed result is used
to populate the \texttt{id_requestor_xsigned} field, and must match exactly
the contents used to populate the preceeding Validation request. The
cross-signed result is good for exactly one set of Validation and Reputation
requests for a particular \texttt{id_subject} and can be destroyed by the caller
after the Reputation request has been submitted.

The \texttt{comms_initiator} field should be populated with a non-zero value
if the device making the reputation amendment request initiated communications
with the other device.

The \texttt{comms_replyreceived} field should be populated with a non-zero value
if any sort of reply was ever received from the other device.

The \texttt{comms_successful} field should be populated with a non-zero value
if all communications with the other device completed successfully.

The \texttt{comms_noreplytmout_s} field should be set to the communication
timeout length. This is the number of seconds after attempting to initiate
communication with another device before the other device's response was
received.

The \texttt{success} field returned from the server will be zero if the
request was successful, and negative otherwise.


\begin{lstlisting}[name=Reputation Request Packing and Response Unpacking Functions]
int ATMIpack_rep_request(const atmi_context_t *ctx,
                         atmi_session_t *ssn,
                         atmi_rep_request_t *rep);

int ATMIunpack_rep_response(const atmi_context_t *ctx,
                            atmi_session_t *ssn,
                            const void *pinbuf, size_t nin,
                            atmi_rep_response_t *rep);
\end{lstlisting}

The pack request and unpack response functions require \texttt{ctx}, a pointer
to the Atonomi library context structure with key data, and \texttt{ssn},
a pointer to a new or current session structure. The pack function will
construct an encrypted message with the data in \texttt{rep}. The unpack function
will read \texttt{nin} bytes of packed, encrypted data from \texttt{pinbuf},
decrypt it, and place the response contents in \texttt{rep}.

