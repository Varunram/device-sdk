%%--------------------------------------------------
%% Intro
\chapter{Introduction to Atonomi}

Atonomi provides a network solution which tackles the question of how to
establish identity, trust, and reputation of IoT devices. The blockchain-based
approach allows companies to securely exchange services and data to, from, and
involving these devices.

\urldef{\atmiwpurl}\url{https://uploads-ssl.webflow.com/5a9f110b6e90d20001b2307d/5aa600a2dc199e000140f98a_Atonomi-Network-White-Paper-v0.9.1.3s%20(1).pdf}

The Atonomi whitepaper \footnote{\atmiwpurl}
provides a more complete description of the Atonomi network. The information
that follows is a brief overview of the important concepts.


\section{Device Identity}
The concept of device identity (sometimes referred to as the identity token)
is essential to securing IoT devices on the Atonomi Network. Based upon the
capabilities of the hardware this identity token could be represented by a
variety of things. Ideally it is a device_id embedded within a secure hardware
element on the device. Device identity will be created during the device or
application development process. Developers need to ensure a unique identifier
is given to each device as the Atonomi Network doesn't allow duplicate
registrations. Identity also incorporates an Elliptic Curve 25519 public/private
key pair created during the either the development or manufacturing process,
as this is required for later device registration with Atonomi. The SDK includes
an EC 25519 key generation tool.


\section{Device Registration}
The Atonomi Identity Registration Service (IRS) is the primary component,
that when globally distributed, becomes the central hub of the Atonomi Identity
Registry Network (IRN). The IRS is a cloud-based, globally accessible, highly
available, high volume service that essentially provides the first step in the
Atonomi process. Devices that have been Atonomi enabled via the Atonomi Embedded
SDK will contact the Identity Registration Service upon first production boot.
Devices will submit to the IRS their identity token. The manufacturer or
developer of the device will have pre-registered that devices identity with the
IRS via the manufacturer GUI. Assuming the IRS finds a match to the devices
identity in its white-list, the device will become activated (within the
Atonomi network).

\textbf{Note:} The Atonomi smart contract includes a device registration
function that writes
new device_ids to the blockchain when the manufacturer or developer adds new
devices to the whitelist via the interface. The Atonomi smart contract also
facilitates the payment in ATMI for the Registration of devices.


\section{Device Activation}
The Atonomi Activation Functionality comes into play after the Atonomi enabled
device has been sold to an end user. The end user will receive activation
instructions that send them to the Atonomi web portal where they will enter the
device identifier and pay for device activation via MetaMask. The Atonomi
Embedded SDK, integrated into the application code by the developer or OEM,
will contact the Activation Service upon first production boot. Assuming the
IRS finds a match to the devices identity in its whitelist, its signature is
verified, and the activation payment cleared, the device will become activated
within the Atonomi network.

\textbf{Note:} The instructions the device owner receives will include a URL
for a device activation portal and the public device_id for the device. The
portal will be a simple web page, backed by the MetaMask plugin that will take
in the device_id and when the Activate `button' is pressed invoke MetaMask to
handle the activation transaction.


\section{Device Validation \&{} Reputation}
Devices that have been Atonomi enabled via the Atonomi Embedded SDK and that
have successfully registered and activated on the Atonomi network can begin to
utilize the Atonomi Validation functionality. Devices will exchange their
signed device identifier with other devices that they want to transact with.
Each device can then call the Atonomi Validation Service, pass in its
counterpart's signed device identifier and receive back an indicator of
whether or not the device is an Atonomi network `member' or not. In the event
that the other device is a member of the network, the Atonomi Validation
Service will also send back the current reputation for that device.

\textbf{Note:} For the purposes of Validation the Embedded SDK will have two
additional methods for validating another devices identity. The first
method will involve producing a CENTRI Protected Sessions handshake package
to be sent to another device for validation. The second method will involve
taking in a signed device identity (from another device via the Protected
Sessions handshake) and submitting it to the Atonomi Validation Service
to be validated. The latter method will return a reputation score if the
other device is valid.



\section{Atonomi Smart Contract}
The Atonomi smart contract includes three basic functions:
Registration, Activation, and Reputation.

The Registration Function of the Atonomi smart contract governs the writing
of a new manufacturer member and their Ethereum address to the blockchain
during manufacturer setup. It also governs the writing of new device_ids to
the blockchain when the manufacturer adds new devices to the whitelist
via the interface. It also facilitates the payment in ATMI for the
Registration of devices.


The Activation Function of the Atonomi smart contract governs the writing
of newly activated device_ids to the blockchain when the device owner boots
the device and activates it via the interface. It also facilitates the payment
in ATMI for the activation of devices.


The Reputation Function of the Atonomi smart contract governs the writing of
updated device reputation scores to the blockchain after the reputation
auditors have processed the reports. It also facilitates the payment in
ATMI for the reputation score updates.

